
%\documentclass[conference]{IEEEtran}
\documentclass[10pt,conference]{IEEEtran}

\input{preamble}


\begin{document}
%\setcopyright{acmcopyright}

\title{When Deep Learning Meets Captcha: A Large Scale Study Showing That Captcha Should Be Abandoned}
\author{
}

\IEEEoverridecommandlockouts \makeatletter\def\@IEEEpubidpullup{9\baselineskip}\makeatother \IEEEpubid{\parbox{\columnwidth}{ }
\hspace{\columnsep}\makebox[\columnwidth]{}}

\maketitle

\begin{abstract}

CAPTCHA is a standard secure mechanism to automated tell computers and humans, and is applied to prevent against malicious bot programs.
Current text-based Captcha mechanism combines many anti-segmentation and anti-recognition features such as sophisticated noisy background,
character rotation, distortion and overlapping. Such mechanism is widely deployed by Google, Microsoft, Baidu, Alibaba. Many attacks on
text-based Captcha have been proposed. However, most of these fine prior works focus on a unique Captcha mechanism, making them have a
limited applicability. In this paper, we demonstrate a novel generic attack on current text-based Captcha. We employ the Generative
Adversarial Networks (GANs) to translate the original Captcha to the regular one. Using the translated regular Captcha to train a
Convolutional Neural Network (CNN) model, our approach is able to accurately identify the Captcha. We thoroughly evaluated our approach
using real-world Captchas, and achieve a success rate of over XX with an average speed of 76ms on an ordinary server (with a 3.2-GHz Inter
Xeon CPU with 100-GB RAM and a TITAN Xp GPU). We discovered that, the Captcha with complex noisy background do not defend against
segmentation under our attack. This is demonstrated by the fact that we are able to remove the complex background from the Captcha image.
Since our attack can identify a widely text-based Captchas, this paper calls the community to revisit the risks of using text-based Captcha
to defend against malicious bots.
\end{abstract}

\begin{IEEEkeywords}
    Captcha, Security, Generic Attack, Text-based
\end{IEEEkeywords}


%TODO:
% Read: A pilot study on the security of pattern screen-lock methods and soft side channel attacks
\section{Introduction}

A CAPTCHA (Completely Automated Public Turing Test to Tell Computers and Human Apart) is a automated test that humans can pass but computer programs cannot~\cite{Von2004Telling}. It provides a effective approach for automatically distinguishing humans from computer systems, and therefore is used to defend against automatic spam, registration or malicious bots~\cite{Von2003CAPTCHA,Tam2008Breaking}.

The most widely deployed CAPTCHA is the so-called text-based scheme~\cite{Yan2008Usability}, which mainly consists of distorted English letters and Arabic numerals. The popularity of this scheme is due to its obvious advantages~\cite{Chellapilla2005Building,Chellapilla2005Computers}: (1) most people around the world can recognize English letters and Arabic numerals; (2) the space of the text-based Captcha are huge so that the brute-force attack can be defeated. Given its pervasive usage, a security breach of the text-based Captcha could lead to serious consequences.

The robustness of text-based Captchas is the significant concern in the research communities. Over the past decade, researchers have uncovered a number of ways to recognize text-based Captchas. Many fine prior researches just focus on attacking an unique Captcha scheme~\cite{Gao2013The,Gao2017Research,Mohamed2014A,Yan2008A}. This limits their applicability. Recently the generic attacks have been proposed by Gao \emph{et al.}~\cite{Gao2016A} and Bursztein \emph{et al.} ~\cite{Bursztein2011Text,Bursztein2014The}. They claim that they can break a wide range of text-based Captchas using a generic method. However, these defeated Captchas possess relative simple noisy background or uniform style. Although the security of text-based Captchas have been proven frail, many companies such as Google, Microsoft and Baidu still use such scheme as current text-based Captcha scheme have more complex background or distorted characters. This makes previous attacks invalid.  Recent studies~\cite{Thomas2013Trafficking,Bursztein2014Easy} also demonstrate that text-based Captcha is still a secure mechanism.

The key factor of previous attacks on text-based Captcha lies in the difficulty of segmenting characters~\cite{Chellapilla2005Computers}. Therefore, the basic design principle of text-based Captchas should be anti-segmentation. To this end, current text-based Captchas with more complex noisy background and distorted characters have been proposed and deployed by many companies. In order to increase the difficulty of finding where each character is, such Captcha shorten the distance between characters. All attacks proposed above were invalid due to cannot successfully segment the characters. It is therefore driving us wondering: is current Captcha scheme as secure as it is expected? This fundamental question precipitated our study.

In this paper, we present a novel generic attack on current text-based Captchas using deep learning technology. Our attack employs a variant of the generative adversarial network (GAN)~\cite{pix2pix2016} to transform the distorted captchas to the regular ones. The former serval layers of the GAN is used to remove the complicated noisy background. The middle layers aim to enlarge the distance between adjacent characters and output the distorted captchas with larger inter-character distance which are translated to regular ones by the last layers. At last, the transformed regular captchas are recognized by a Convolutional Neural Network (CNN).

We thoroughly evaluate our approach using real-world captchas collected from some websites. We show that our approach is effective in recognize current text-based captchas and as a results, we can defeat almost all current text-based captchas with a success rate range from XX\% to XX\%. We demonstrate that, the sophisticated noisy background cannot offer stronger protection in term of anti-segmentation under our attack. Our finding suggests that text-based captchas are insecure under the age of artificial intelligence.

\textbf{Contributions} This paper makes the following specific contributions:



In order to design a more security Captcha scheme, currently Captchas deployed by many companies 
\input{background}
\input{generator}
\input{overview}
\input{details}
\section{Experimental Setup}
\subsection{Data Preparation and Collection}
The Captchas used in our evaluation are made up of both training data and testing data, and they come from two different sources. The training Captchas are synthesized by our Captcha generator (Section~\ref{section: Captcha_generator}) and the testing Captchas are collected from corresponding websites using a traditional mining method written through a python script.

\noindent \textbf{Target Websites} We select the target websites based on the following reasons: (1) The Captchas of the websites combines at least two anti-segmentation and three anti-recognition features; (2) the websites must have a large number of active users. According to the two factors, we target ? famous websites for preparing our data: Baidu, XX, XX.

\noindent \textbf{Synthesize Training Captchas.} To ensure the synthesized Captcha as likely as the true one, we first manually analyze the traits of the true Captchas and aggregates these traits to a feature tuple as described in Section~\ref{section: Captcha_generator}. According to the feature tuple, a large number of training Captchas can be synthesized by our generator. In our experiment, we totally synthesized ? kinds of Captchas come from the target websites. For each Captcha scheme, we mined 20000 Captchas to used for training process.

\noindent \textbf{Mine Testing Captchas.} The testing Captchas used in our evalustion are automatically mined from the target websites using a python script. For each target website, we apply 2000 Captchas to test the efficiency of our attacking system.  We recruited nine participators from our institution to manually mark the labels of the mined Captchas. To decrease the marking error rate, the nine participator are divided into three groups and each group has three people. We choose these Captcha which two participators mark right at the same time as the testing data.

\noindent \textbf{Implementation.} Our prototyped attacking system built upon a variant of \emph{Pix2Pix} framework~\cite{Pix2PixCode} in Tensorflow. The developed software ran on an ordinary
server with a 3.2-GHz Inter Xeon CPU with 100-GB RAM and a TITAN Xp GPU. The operating system is Ubuntu 16.04.

\begin{table*}
  \centering
  \caption{Target text-based Captcha schemes used in our experiment}
  \label{Captcha_show}
  \small
  \begin{tabular}{|c|c|c|c|c|}
    \hline
     &  &  & \multicolumn{2}{|c|}{Security Features} \\
     \cline{4-5}
    \multirow{-2}{*}{Scheme} & \multirow{-2}{*}{Website} & \multirow{-2}{*}{Sample} & Anti-segmentation Features & Anti-recognition Features \\
    \hline
    Baidu, Tecent & baidu.com & \includegraphics[width=0.1\textwidth]{fig/experiment_captchas/baidu1.jpg} \includegraphics[width=0.1\textwidth]{fig/experiment_captchas/baidu2.jpg} & \tabincell{c}{Connecting Lines, Overlapping, \\ Only Enligh letters used} & \tabincell{c}{Both hollow and solid characters, \\ varied font size, color, \\ rotating, disortion and Waving used} \\
    \hline
    Alipay & alipay.com & \includegraphics[width=0.1\textwidth]{fig/experiment_captchas/alipay1.jpg} \includegraphics[width=0.1\textwidth]{fig/experiment_captchas/alipay2.jpg} & \tabincell{c}{Overlapping characters used} & \tabincell{c}{Both English letters and \\ Arabic numerals, \\ rotating and distortion used} \\
    \hline
    NetEase & 163.com & \includegraphics[width=0.1\textwidth]{fig/experiment_captchas/netease1.jpg} \includegraphics[width=0.1\textwidth]{fig/experiment_captchas/netease2.jpg} & \tabincell{c}{Complex background, \\ connecting lines} & \tabincell{c}{Both hallow and solid characters, \\ varied rotating angles used} \\
    \hline
    VIPSHOP & vip.com & \includegraphics[width=0.1\textwidth]{fig/experiment_captchas/vipshop1.jpg} \includegraphics[width=0.1\textwidth]{fig/experiment_captchas/vipshop2.jpg} & \tabincell{c}{Complex background, \\ Overlapping characters} & \tabincell{c}{Both English letters \\ and Arabic numerals, \\ varied font colors used} \\
    \hline
     TOUP& tuniu.com & \includegraphics[width=0.1\textwidth]{fig/experiment_captchas/tuniu1.jpg} \includegraphics[width=0.1\textwidth]{fig/experiment_captchas/tuniu2.jpg} & \tabincell{c}{Overlapping characters used} & \tabincell{c}{Both English letters \\ and Arabic numerals, \\ waving characters used} \\
    \hline
  \end{tabular}
\end{table*}

\subsection{Captchas Show}

\section{Experimental Results}
In this section, we first present the overall success rate for breaking the Captchas collected from the target websites. Our results shows that the overall success rates range from XX to XX.
We then analyze how the success rate is affected by the number of the training Captchas.
Finally, we demonstrate the risks of our attacking approach by simulating a real world attacking process before evaluating the robustness of the secure features using variant types of text-based Captchas.

\begin{table}[t]
    \centering
    \caption{The overall success rate and average attack speeds for each Captcha scheme.}
    \label{table: feature_number}
    \begin{tabular}{lcc}
        \toprule
        Scheme & Success rate  & Speed (ms)\\
        \midrule
        Baidu & 67\% & 70 \\
        Alipay & 86\% & 72 \\
        NetEase & 45\% & 74 \\
        VIPSHOP & 90\% & 65 \\
        TOUP & 69\% & 50 \\
        \bottomrule
    \end{tabular}
\end{table}

\subsection{Overall Success Rate}
In this experiment, the training data were synthesized used our Captcha generator, and the number of each training Captcha scheme is 20000. The corresponding testing data were collected from the target websites and its number is 2000.
\section{Related Work} 

The Automated Turing Tests were first proposed by Naor~\cite{Naor1996Verification}, but they did not provide a formal definition. Lillibridge \emph{et al.}~\cite{Lillibridge2001Method} developed the first practical Automated Turing Test to prevent bots from automatically registering web pages. This system was effective for a while and then was defeated by common Optical Character Recognition (OCR) technology.

Text-based Captchas based on English letters and Arabic numerals, is still the most widely deployed scheme. Many research communities focus on developing attack approaches for existing text-based Captchas, and then explore guidelines for better designs. 
\input{conclusion}
%\input{materials}
%\section{Related Work} 

The Automated Turing Tests were first proposed by Naor~\cite{Naor1996Verification}, but they did not provide a formal definition. Lillibridge \emph{et al.}~\cite{Lillibridge2001Method} developed the first practical Automated Turing Test to prevent bots from automatically registering web pages. This system was effective for a while and then was defeated by common Optical Character Recognition (OCR) technology.

Text-based Captchas based on English letters and Arabic numerals, is still the most widely deployed scheme. Many research communities focus on developing attack approaches for existing text-based Captchas, and then explore guidelines for better designs. 

%\section*{Acknowledgements}
%    We would like to thank all participants who help for completing the experiments. Thank all volunteers for their time and insights as
%    well as the anonymous reviewer for their critical and constructive comments. This work was supported by NSFC (Grant No. 61672427) and
%    the UK Engineering and Physical Sciences Research Council (Grants No. EP/M01567X/1(SANDeRs) and EP/M015793/1(DIVIDEND)).

%\begin{spacing}{0.98}
\bibliographystyle{IEEEtranS}
\balance
\bibliography{refs}
%\end{spacing}

\end{document}
