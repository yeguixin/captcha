\section{Materials}

1. CPATCHAs are computer-generated tests that humans can pass but current computer systems cannot. CAPTCHAs provide a method for automatically distinguishing a human from a computer program, and therefore can protect Web services from abuse by so-called "bots".

most CAPTCHAs consist of distorted images, usually text, for which a user must provide some description.

CAPTCHAs can distinguish between humans and computers with high probability, they are used for many different security applications: they prevent bots from voting continuously in online polls, automatically registering for millions of spam email accounts, automatically purchasing tickets to buy out an event, etc~\cite{Tam2008Breaking}.

2. We carry out a systematic study of existing visual CAPTCHAs based on distorted characters.

For example, Gmail improves its service by blocking access to automated sapmmers, eBay improves its marketplace by blocking bots from flooding the site with scams, and Facebook limits creation of fraudulent profiles used to spam honest users or cheat at games. The most widely used CAPTCHA schemes use combinations of distorted characters and obfuscation techniques that human can recognize but that may be difficult for automated scripts.

In fact, as we substantiate by thorough study, many popular websites still rely on schemes that are vulnerable to automated attacks.

3. The abuse of the resources of online services using automated means, such as denial-of-service or password dictionary attacks, is a common security problem~\cite{Mohamed2014A}.

4. Since their inception, captchas have been widely used for preventing fraudsters from performing illicit actions.

In this paper, we conduct a comprehensive study of reCaptcha, and explore how the risk analysis process is influenced by each aspect of the request.

1. Captcha is a security mechanism designed to differentiate between computers and humans, and is used to defend against malicious bot programs.

We propose a simple but an effective method to attack the two-layer Captcha deployed by Microsoft, and achieve a success rate of 44.6\% with an average speed of 9.05s on a standard desktop computer (with a 3.3-GHz Intel Core i3 CPU and 2-GB RAM), thus demonstrating clear security issues.

Researchers have recently claimed that their simple generic attacks have broken a wide range of text-based Captchas in a single step~\cite{Bursztein2014The,Gao2016A}. Algorithm many text-based Captchas have proven insecure, the most recent studies~\cite{Thomas2013Trafficking,Bursztein2014Easy} suggest that Captcha is still a usefull security mechanism, and the security of text-based Captchas is currently a hot topic in the academic field.~\cite{Gao2017Research}

2. CAPTCHA is now a standard security technology for differentiating between computers and humans, and the most widely deployed schemes are text-based.

A main feature of such schemes is to use contour lines to form connected hollow characters with the aim of improving security and usability simultaneously, as it is hard for standard techniques to segment and recognize such connected characters, which are however easy to human eyes.

We show that with a simple but novel attack, we can successfully break a whole family of hollow CAPTCHAs, including those deployed by all the major companies. While our attack casts serious doubt on the viability of current designs, we offer lessons and guidelines for designing better hollow CAPTCHAs~\cite{Gao2013The}.

3. CAPTCHAs are automated tests to tell computers and humans apart. They are designed to be easily solvable by humans, but unsolvable by machines~\cite{Stark2015CAPTCHA}.

4. A CAPTCHA (Completely Automated Public Turing Test to Tell Computers and Human Apart) is a program that generates and grades tests that are human solvable, but intend to be beyond the capabilities of current computer programs~\cite{Von2004Telling}. This technology is now almost a standard security mechanism for defending against undesirable or malicious Internet bot programs, such as those spreading junk emails and those grabbing thousands of free email accounts instantly. It had found widespread application on numerous commercial web sites including Google, Yahoo, and Microsoft's MSN.

\textbf{Background} The most widely CAPTCHAs are the so-called text-based schemes, which rely on sophisticated distortion of text images aimed at rendering them unrecognisable to the state of the art of pattern recognition methods. The popularity of such schemes is due to the fact that they have many advantages~\cite{Chellapilla2005Building}, for example, being intuitive to users world-wide (the user task performed being just character recognition), having little localization issues (people in different countries all recognise Roman characters), and of good potential to provide strong security (e.g. the space a brute force attack has to search can be huge, if the scheme is properly designed).~\cite{Yan2008A} 